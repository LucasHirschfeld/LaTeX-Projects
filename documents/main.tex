\documentclass[12pt,a4paper]{article}

% Pakete
\usepackage[utf8]{inputenc}
\usepackage[T1]{fontenc}
\usepackage[ngerman]{babel}
\usepackage{amsmath,amssymb,amsfonts}
\usepackage{graphicx}
\usepackage{hyperref}
\usepackage{geometry}
\usepackage{biblatex}

% Seitenränder
\geometry{
    left=2.5cm,
    right=2.5cm,
    top=2.5cm,
    bottom=2.5cm
}

% Bibliographie
\addbibresource{../bibliography/references.bib}

% Metadaten
\title{Mein LaTeX-Dokument}
\author{Lucas Hirschfeld}
\date{\today}

\begin{document}

\maketitle

\tableofcontents
\newpage

\section{Einleitung}

Dies ist ein Beispiel-LaTeX-Dokument, das automatisch mit GitHub Actions kompiliert wird.

\section{Mathematik}

Eine einfache Gleichung:
\begin{equation}
    E = mc^2
\end{equation}

Ein komplexeres Beispiel:
\begin{align}
    \nabla \times \mathbf{E} &= -\frac{\partial \mathbf{B}}{\partial t} \\
    \nabla \times \mathbf{B} &= \mu_0 \mathbf{J} + \mu_0 \epsilon_0 \frac{\partial \mathbf{E}}{\partial t}
\end{align}

\section{Grafiken}

% Beispiel für Grafikeinbindung
% \begin{figure}[htbp]
%     \centering
%     \includegraphics[width=0.8\textwidth]{../figures/beispiel.png}
%     \caption{Beispielgrafik}
%     \label{fig:beispiel}
% \end{figure}

\section{Literatur}

Zitationen funktionieren mit \texttt{biblatex} \cite{beispielquelle}.

\newpage
\printbibliography

\end{document}
