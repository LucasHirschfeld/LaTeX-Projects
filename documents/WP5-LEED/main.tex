\documentclass[fontsize=11pt,reqno,a4paper,oneside]{scrartcl}
\usepackage[utf8]{inputenc}
\usepackage[T1]{fontenc}
\usepackage[onehalfspacing]{setspace}
\usepackage{microtype}
\usepackage[english]{babel}
\usepackage{geometry}
\geometry{top=2cm, bottom=2cm, left=2.5cm, right=2.5cm}

\usepackage{amsmath, amsthm, amssymb, mathtools}
\usepackage{xcolor}
\usepackage{graphicx}
\usepackage{float}
\usepackage{multirow}
\usepackage{siunitx}
\usepackage{textpos}
\usepackage[pscoord]{eso-pic}
\newcommand{\placetextbox}[3]{% \placetextbox{<horizontal pos>}{<vertical pos>}{<stuff>}
  \setbox0=\hbox{#3}% Put <stuff> in a box
  \AddToShipoutPictureFG*{% Add <stuff> to current page foreground
    \put(\LenToUnit{#1\paperwidth},\LenToUnit{#2\paperheight}){\vtop{{\null}\makebox[0pt][c]{#3}}}%
  }%
}%
\newcommand{\angstrom}{\text{\normalfont\AA}}


\usepackage[backend=biber,
	citestyle=numeric-comp,
	bibstyle=numeric,
	bibwarn=true,
	autocite=superscript,
	abbreviate=true, 
	sorting=none, 
	url=false, 
	doi=false,
	eprint=false,
	isbn=false]{biblatex}
\addbibresource{literature.bib}
\usepackage{csquotes}
	\DeclareCiteCommand{\supercite}[\mkbibsuperscript]%
	{\usebibmacro{cite:init}%
		\let\multicitedelim\supercitedelim%
		\iffieldundef{prenote}%
		{}%
		{\BibliographyWarning{Ignoring prenote argument}}%
		\iffieldundef{postnote}%
		{}%
		{\BibliographyWarning{Ignoring postnote argument}}%
		\bibleftbracket% HIER DIE OEFFNENDE KLAMMER
	}%
	{\usebibmacro{citeindex}%
		\usebibmacro{cite:comp}}
	{}%
	{\usebibmacro{cite:dump}%
		\bibrightbracket% HIER DIE SCHLIESSENDE KLAMMER
	}
\addbibresource{References.bib}
\usepackage{csquotes}



\usepackage[version=4]{mhchem}
\usepackage{chemformula}
\usepackage{hyperref}
\hypersetup{colorlinks=true, linkcolor=black, citecolor=black, urlcolor=black}

\usepackage[headsepline]{scrlayer-scrpage}
\automark[section]{section}

\setlength{\parindent}{0pt}

\def\Autor{Lucas Hirschfeld, s6luhirs@uni-bonn.de}
\def\Titel{Low Energy Electron Diffraction (LEED)}
\def\Date{First Submission: xx.xx.xxxx}
\def\PiC{Supervisor: Morris Mühlpointner}
\def\Abstract{ } 


\begin{document}
\pagestyle{empty}
%\pagenumbering{empty}
\begin{titlepage}
    \begin{center}
        \vspace*{\fill}
        {\LARGE\bfseries \Titel}\par\vspace{2cm}
        {\Autor\par\vspace{0.5cm}\PiC\par\vspace{1cm}\par\vspace{0.2cm}\Date\par\vspace{5cm}}
    
    \end{center}
        {\Abstract\vspace{0.5cm}\par}\vfill
\end{titlepage}


\pagestyle{scrheadings}
\ihead{\headmark}
\ofoot{\pagemark}
\tableofcontents
\pagenumbering{arabic}
\clearpage

\newpage
\section{Introduction}

Low Energy Electron Diffraction (LEED) is a cornerstone technique in surface science, enabling the investigation of atomic arrangements at crystalline surfaces with high sensitivity and precision \cite{vanHove1986, Henzler1994}. By directing low-energy electrons (typically 20--300~eV) onto a surface, LEED provides diffraction patterns that directly reflect the symmetry and periodicity of the outermost atomic layers \cite{Pendry1974}. This makes LEED particularly valuable for studying two-dimensional surface structures, reconstructions, and adsorbate-induced modifications.

The LEED experiment utilizes an electron gun, focusing and deflection electrodes, and a fluorescent screen to visualize the resulting diffraction pattern. By varying the electron energy, the wavelength of the electrons is tuned, allowing for detailed analysis of the diffraction spots and extraction of structural parameters such as lattice constants and surface symmetries \cite{Oura2013}.

A fundamental aspect of LEED is its demonstration of the wave nature of electrons, as first observed by Davisson and Germer in 1927 \cite{Davisson1927}. Their experiment provided crucial evidence for de Broglie's hypothesis, laying the groundwork for modern quantum mechanics. In LEED, the constructive and destructive interference of elastically scattered electrons forms a direct analogy to optical diffraction, making it an essential tool for both research and teaching in surface physics.

\newpage
\section{Experimental}


\newpage
\section{Results}



\newpage
\section{Discussion}

\newpage
\printbibliography[heading=bibintoc,
title={References}] 


\end{document}
