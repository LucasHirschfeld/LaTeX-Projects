\chapter{Summary and Outlook}

In summary, the present study has expanded the understanding of the phase behaviour of \ac{QA} on Ag(100) by examining the $\alpha$- and the $\beta$-phase in greater detail. The investigation of the $\alpha$- and the $\beta$-phase was conducted by utilizing \ac{XPS} for the C1s, N1s and O1s components.

The results for the $\alpha$-phase are consistent with the prior findings by N. Humberg.\autocite{Humberg2024} The proposed structure model has been be confirmed. In addition, all \ac{XPS} spectra, particularly those of the C1s core-level, can be subdivided into multiple peaks and allocated to particular atoms of \ac{QA}. Furthermore, the consequences of beam damage due to exposure of \ac{QA} to the x-ray were investigated. The results demonstrated that the deprotonation of the amine group occured as a result of this exposure.

A subsequent examination of the results for the $\beta$-phase reveals inconsistencies with the proposed structure model by N. Humberg\autocite{Humberg2024} for this phase. It was ascertained that the $\beta$-phase is composed of two distinct oxygen atoms, exhibiting a stoichiometric ratio of 1~:~1. A comparable observation was made in the case of the nitrogen atoms. Additionally, it was determined that deprotonation of one of the nitrogen atoms is necessary due to its shift in binding energy, leading to only one hydrogen bond per molecule. The findings of this study have led to the proposition of the novel structure model in \autoref{fig:models}.

Nevertheless, the extant structure models leave unresolved questions in need of further investigation. To further refine our comprehension of \ac{QA} on Ag(100), it is imperative to undertake additional research to develop a more detailed structure model. Therefore, \ac{NIXSW} measurements could be used to determine the adsorption height of the atoms, which allow a revision of the structure model. Beyond that, different \ac{NIXSW} measurements could be used for triangulation of the $\beta$-phase. This should give exact adsorption sites for the atoms and allow a detailed structure model.

\cleardoublepage