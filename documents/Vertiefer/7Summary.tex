\chapter{Summary and Outlook}

This study investigated the electronic properties and molecular orientation of \ac{QA} in different phases adsorbed on the Ag(100) surface using \ac{XPS}. The primary research question addresses how the \ac{QA} molecules interact with the substrate and what changes occur upon phase transition. Specifically, the aim was to understand the intermolecular interactions via hydrogen bonds and the interactions with the metal surface

The \ac{XPS} measurements revealed distinct peaks for the carbon, nitrogen and oxygen atoms with different chemical environments within \ac{QA}. The photoemission lines for C1s show different peaks for aromatic carbon atoms (C$_{\text{arom}}$), carbon atoms bonded to nitrogen (C$_{\text{NH}}$), carbon atoms in carbonyl groups (C$_{\text{CO}}$) and carbon atoms bonded to oxygen (C$_{\text{O}}$). All these peaks were observed in both the $\alpha$- and $\beta$-phase. Furthermore, the different peaks in the C1s \ac{XPS} spectra exhibit a shift towards lower \acp{BE} between the two phases, indicating an increased interaction with the substrate in the $\beta$-phase.

The N1s and O1s \ac{XPS} spectra show one peak for the $\alpha$-phase, whereas the $\beta$-phase exhibits two distinct peaks for the nitrogen and oxygen atoms with an area ratio of 1~:~1. This indicates a change in the number of hydrogen bonds per molecule between the two phases, with the $\alpha$-phase having two hydrogen bonds per molecule and the $\beta$-phase having one hydrogen bond per molecule. Additionally, the consequence of beam damage due to exposure of \ac{QA} to the x-ray was investigated, demonstrating the deprotonation of the amine group as a result of this exposure.

As already stated, the N1s \ac{XPS} spectra show a peak for the deprotonated nitrogen atom in the $\beta$-phase, which is not present in the $\alpha$-phase. This indicates that the deprotonation of one nitrogen atom occurs during the phase transition from the $\alpha$- to the $\beta$-phase, explaining the change in the number of hydrogen bonds per molecule.

The results for the $\alpha$-phase are consistent with the prior findings by N. Humberg.\autocite{Humberg2024,Humberg2020} The proposed structure model has been confirmed and the \ac{XPS} spectra for the C1s, N1s and O1s components can be subdivided into multiple peaks, which can be allocated to particular atoms of \ac{QA}.

The subsequent examination of the results for the $\beta$-phase reveals inconsistencies with the proposed structure model by N. Humberg\autocite{Humberg2024,Humberg2020}. It was ascertained that the $\beta$-phase is composed of two distinct oxygen atoms, exhibiting a stoichiometric ratio of 1~:~1. A similar observation was made in the case of the nitrogen atoms. The observed shifts in the \ac{BE} are indicative of the deprotonation of one of the nitrogen atoms. This results in the formation of a single hydrogen bond per molecule. The findings of this study have led to the proposition of a novel structure model in \autoref{fig:models}.

Nevertheless, the extant structure models leave unresolved questions in need of further investigation. To further refine our comprehension of \ac{QA} on Ag(100), it is imperative to undertake additional research to develop a more detailed structure model. Therefore, \ac{NIXSW} measurements could be used to determine the adsorption height of the atoms, which allow a revision of the structure model. Beyond that, different \ac{NIXSW} measurements could be used for triangulation of the $\beta$-phase. This should give exact adsorption sites for the atoms and allow a detailed structure model.

\cleardoublepage