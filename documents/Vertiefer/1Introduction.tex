\pagenumbering{arabic}
\chapter{Introduction}

The examination of surfaces and adsorbates is essential for endless disciplines, each offering unique insights and applications. A broad array of disciplines is encompasses within these fields, which include but are not limited to: coatings, electronic, nanotechnology, energy and transportation.\autocite{UKResearchandInnovation2023} Furthermore, the relevance of surface science to environmental degradation and corrosion has been well-documented.\autocite{TheInternationalUnionforVacuumScienceTechniqueandApplications2023}

In order to be applicable in these fields, it is necessary that the surfaces and the adsorbates possess particular characteristics, such as well-defined structure or a high dipole moment. For instance, the utilization of specific organic adsorbates has facilitated the development of innovative organic solar cells, which exhibit a high degree of flexiblity.\autocite{Abdulrazzaq2013,Chamberlain1983,Green2010}

One molecule that merits particular attention in the context of these applications is \ac{QA}. It has been established that the molecule in question is capable of forming one-dimensional (1D) molecular chains within its bulk crystal structure\autocite{Paulus2007} and on surfaces\autocite{Wagner2014, Eberle2019, Trixler2007, Humberg2024}, with these chains being connected via intermolecular hydrogen bonds.

In previous studies, the potential of \ac{QA} for application in electronic devices has been demonstrated.\autocite{Glowacki2013,DanielGlowacki2012, Wang2016} Its potential extends beyond the domain of printer toners, where it is generally employed, offering significant promise for applications in diverse fields. Examples of applications for \ac{QA} are \acp{OFET}\autocite{Jeon2018, Jeong2017, Kanbur2019, Berg2009}, \acp{OLED}\autocite{Wang2016a, Min2021, Cunha2018} and \acp{OPV}\autocite{Dunst2017, Sung2017}.
The subsequent exploration of the structural characteristics of \ac{QA} on metal surfaces, such as \ac{Ag}, is driven by the promising properties. This work represents a continuation of the previous examination conducted by N. Humberg, which investigated the different structures of \ac{QA} on the Ag(100) surface.\autocite{Humberg2024}

The utilization of alternative techniques, specifically \ac{XPS}, in comparison to previous methods will facilitate a more detailed comprehension of the structure of \ac{QA} on the Ag(100) surface. This, in turn, will lead to the identification of additional properties inherent to this adsorbate-substrate system.

\cleardoublepage