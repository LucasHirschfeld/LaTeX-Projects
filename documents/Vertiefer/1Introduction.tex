\pagenumbering{arabic}
\chapter{Introduction}

The interface of organic semiconductors with metal surfaces is of particular importance within the domain of organic electronic devices. The aforementioned electronic devices include, for example, \acp{OLED},\autocite{Wang2012, Facchetti2007} \acp{OFET}\autocite{Kulkarni2004, Shahnawaz2019} and \acp{OPV}.\autocite{Hains2010} A significant number of investigations into these interfaces have focused on the structural characteristics of the organic semiconductors, which are imperative for the functionality of the devices. Consequently, organic molecules with a large $\pi$-conjugtated system are frequently utilized, as they demonstrate high electron and hole mobilities.

Moreover, the property of 1D molecular chain formation is of significance for atomic and molecular wires. These wires are important components for modern devices, including those employed in computing, energy storage, photonics, photovoltaics and other applications.\autocite{Goktas2018, Zhou2019, Jia2019, Daniels2023, Gu2023, Patzsch2017, Imran2018} In addition, 1D structures are of considerable interest for the investigation of the physical properties of organic adsorbates. A comparison between 1D structures and structures with higher dimensionality can provide insights into the relation between individual atoms or molecules and the corresponding properties of the system.

One molecule that has been extensively studied in the context of electronic devices is \ac{QA}. \Ac{QA} is an organic, prochiral molecule with a large $\pi$-conjugated system, which has been shown to form well-defined structures on metal surfaces. In previous studies, the potential of \ac{QA} for application in electronic devices has been demonstrated.\autocite{Glowacki2013,DanielGlowacki2012, Wang2016} Its potential extends beyond the domain of printer toners, where it is generally employed, offering significant promise for applications in diverse fields. Examples of applications for \ac{QA} are \acp{OFET}\autocite{Jeon2018, Jeong2017, Kanbur2019, Berg2009}, \acp{OLED}\autocite{Wang2016a, Min2021, Cunha2018} and \acp{OPV}\autocite{Dunst2017, Sung2017}.

The well-defined structures of \ac{QA} consist of 1D molecular chains. These structures have been identified in the bulk crystal structure\autocite{Paulus2007} and on surfaces.\autocite{Humberg2024, Wagner2014, Trixler2007, Eberle2019} The formation of the 1D molecular chains is driven by intermolecular hydrogen bonds and the interaction with the adsorbate.\autocite{Humberg2024, Humberg2020}

For \ac{QA}, two distinct phases on Ag(100) have been identified. The initial phase is predominantly attributable to intermolecular hydrogen bonds, resulting in a 1D molecular chain structure that is not commensurate. The second phase is characterized by a decreased number of hydrogen bonds and is commensurate. The phase transition between these two phases is driven by the interaction with the substrate and the temperature.\autocite{Humberg2024, Humberg2020} The structure formation of \ac{QA} on Ag(100) in the distinct phases show the competition between the intermolecular hydrogen bonds and the interaction with the substrate.

As a result of the aforementioned phase behaviour of \ac{QA} on Ag(100), the molecule is of particular interest for the investigation of the structural characteristics. The formation of 1D molecular chains with intermolecular hydrogen bonds and the phase transition between two distinct phases provide a unique opportunity to study the interplay between intermolecular interactions and substrate interactions.

The objective of this research is to ascertain how the structure of \ac{QA} on Ag(100) is impacted by the interplay between intermolecular interactions via hydrogen bonds and substrate interactions of the molecule across the two distinct phases. Therefore, \ac{XPS} is employed as the primary technique, which provides insights into the electronic structure and chemical environment of the adsorbate. The investigation is conducted for both phases which allows the observation of the differences in these phases.

The measured \ac{XPS} spectra provide insight into the number of hydrogen bonds and the interaction with the substrate. The ratio of hydrogen bonds per molecule between the two phases has been determined as 2:1, contradicting the earlier report by N. Humberg, who stated a ratio of 2:1.5.\autocite{Humberg2024, Humberg2020} This prompts the necessity for a novel structure model, a subject that will be examined in the subsequent chapters.

\cleardoublepage